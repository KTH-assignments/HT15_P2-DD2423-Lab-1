\documentclass[12pt]{article}
\usepackage{amsmath}
\usepackage{lscape}
\usepackage{graphicx}
\usepackage{pgfplots} 


\title{DD$2423$ - Lab I}
\date{}

\begin{document}
	\maketitle
  
  	\section{Question 1}
  		What we see here is that:
  
  		\begin{itemize}
 
  		\item The further the non-zero point $(p,q)$ from the origin $O(0,0)$, the smaller the wavelength of the spatial image 
  		(more dense lines in the real and imaginary part of the spatial image),
  
  		\item The amplitude of all the spatial images is the same,
  
  		\item The direction of the waveforms in the spatial images is dictated by the position of the non-zero point $(p,q)$ relative to the origin $O(0,0)$
  
  		\end{itemize}


  	\section{Question 2}
  
  		We exploit equation $4.2-33$ from \cite{GZ}  
  
  		\begin{equation}
  			\sum_{x=0}^{M-1}\sum_{y=0}^{N-1} s(x,y) \cdot A \delta(x-x_0, y-y_0) = A \cdot s(x_0, y_0)
  		\end{equation}
  
  		knowing that the output Fourier transform is a delta function at $(p,q)$. Hence, for a quadratic image $M=N$ and in the spatial domain:
  
  		\begin{equation}
  			f(x,y)= \frac{1}{N^2} \sum_{x=0}^{N-1}\sum_{y=0}^{N-1} \delta(u - p, v - q) \cdot e^{\dfrac{2 \pi i \cdot (xu + yv)}{N}} = 
  			\frac{1}{N^2} \cdot e^{\dfrac{2 \pi i \cdot (px + qy)}{N}}
  		\end{equation}
  
	  	Hence, 
	  	\begin{equation}
	  		f(x,y) = \frac{1}{N^2} \cdot (\cos(\dfrac{2 \pi \cdot (px + qy)}{N}) + i\sin(\dfrac{2 \pi \cdot (px + qy)}{N}))
	  		\label{eq:q2:sin_form}
	  	\end{equation}
  
  TODO: add figures
  
	  	\begin{figure}[!htb]
			\centering
			\scalebox{.5}{% This file was created by matlab2tikz.
%
%The latest updates can be retrieved from
%  http://www.mathworks.com/matlabcentral/fileexchange/22022-matlab2tikz-matlab2tikz
%where you can also make suggestions and rate matlab2tikz.
%
\definecolor{mycolor1}{rgb}{0.00000,0.44700,0.74100}%
%
\begin{tikzpicture}

\begin{axis}[%
width=4.521in,
height=3.566in,
at={(0.758in,0.481in)},
scale only axis,
xmin=0,
xmax=140,
ymin=-0.008,
ymax=0.008,
axis background/.style={fill=white}
]
\addplot [color=mycolor1,solid,forget plot]
  table[row sep=crcr]{%
1	0.00195748866460033\\
2	0.0037366152086228\\
3	0.00529177840245197\\
4	0.00652976566209381\\
5	0.00737637513408318\\
6	0.00778086316860173\\
7	0.00771898576666483\\
8	0.00719445170465574\\
9	0.00623870023956485\\
10	0.0049090167187246\\
11	0.00328509903966998\\
12	0.00146428075791393\\
13	-0.00044430284241422\\
14	-0.00232625604396381\\
15	-0.00406877928874101\\
16	-0.00556743010089505\\
17	-0.0067323831069475\\
18	-0.00749381394354109\\
19	-0.00780608435477231\\
20	-0.00765047763486313\\
21	-0.0070363204605939\\
22	-0.00600042387367352\\
23	-0.00460487691916207\\
24	-0.00293332518372927\\
25	-0.00108595728883801\\
26	0.000826500164114749\\
27	0.00268941926876141\\
28	0.00439114132316963\\
29	0.00582966937257579\\
30	0.00691878165120943\\
31	0.00759319949882841\\
32	0.0078125\\
33	0.00756353883233632\\
34	0.0068612381042336\\
35	0.00574769196109545\\
36	0.00428964356776166\\
37	0.00257448469049206\\
38	0.000705017653534714\\
39	-0.00120670637452706\\
40	-0.00304610344697526\\
41	-0.00470292471353221\\
42	-0.00607786446011923\\
43	-0.00708851224445987\\
44	-0.00767429237143736\\
45	-0.00780009464843351\\
46	-0.00745837880227468\\
47	-0.00666962642430676\\
48	-0.00548111335514479\\
49	-0.00396407608927815\\
50	-0.00220944203813713\\
51	-0.000322379568951686\\
52	0.00158400552358809\\
53	0.00339544929517815\\
54	0.00500337834633231\\
55	0.00631141743982026\\
56	0.00724116599083312\\
57	0.00773689720153487\\
58	0.00776889818565135\\
59	0.00733525088440154\\
60	0.00646194703013342\\
61	0.00520133026703267\\
62	0.00362895880428341\\
63	0.00183907664638817\\
64	-6.10351562499502e-05\\
65	-0.00195748866460029\\
66	-0.00373661520862282\\
67	-0.00529177840245198\\
68	-0.00652976566209382\\
69	-0.00737637513408317\\
70	-0.00778086316860173\\
71	-0.00771898576666483\\
72	-0.00719445170465573\\
73	-0.00623870023956483\\
74	-0.00490901671872463\\
75	-0.00328509903967001\\
76	-0.00146428075791396\\
77	0.000444302842414243\\
78	0.00232625604396383\\
79	0.00406877928874099\\
80	0.00556743010089503\\
81	0.00673238310694748\\
82	0.0074938139435411\\
83	0.00780608435477231\\
84	0.00765047763486312\\
85	0.00703632046059392\\
86	0.00600042387367354\\
87	0.0046048769191621\\
88	0.00293332518372925\\
89	0.00108595728883799\\
90	-0.000826500164114717\\
91	-0.00268941926876138\\
92	-0.0043911413231696\\
93	-0.00582966937257581\\
94	-0.00691878165120944\\
95	-0.00759319949882842\\
96	-0.0078125\\
97	-0.00756353883233634\\
98	-0.00686123810423365\\
99	-0.00574769196109544\\
100	-0.00428964356776164\\
101	-0.00257448469049204\\
102	-0.000705017653534691\\
103	0.00120670637452697\\
104	0.00304610344697529\\
105	0.00470292471353223\\
106	0.00607786446011925\\
107	0.00708851224445988\\
108	0.00767429237143734\\
109	0.00780009464843351\\
110	0.00745837880227467\\
111	0.00666962642430674\\
112	0.00548111335514477\\
113	0.00396407608927813\\
114	0.00220944203813722\\
115	0.000322379568951663\\
116	-0.00158400552358811\\
117	-0.00339544929517817\\
118	-0.00500337834633232\\
119	-0.00631141743982021\\
120	-0.00724116599083313\\
121	-0.00773689720153487\\
122	-0.00776889818565135\\
123	-0.00733525088440153\\
124	-0.00646194703013347\\
125	-0.00520133026703274\\
126	-0.00362895880428339\\
127	-0.00183907664638814\\
128	6.10351562499732e-05\\
};
\end{axis}
\end{tikzpicture}%}
			\caption{}
			\label{fig:}
	  	\end{figure}

	  	\begin{figure}[!htb]
			\centering
			%\scalebox{.5}{% This file was created by matlab2tikz.
%
%The latest updates can be retrieved from
%  http://www.mathworks.com/matlabcentral/fileexchange/22022-matlab2tikz-matlab2tikz
%where you can also make suggestions and rate matlab2tikz.
%
\begin{tikzpicture}

\begin{axis}[%
width=3.566in,
height=3.566in,
at={(1.236in,0.481in)},
scale only axis,
axis on top,
xmin=0.5,
xmax=128.5,
y dir=reverse,
ymin=0.5,
ymax=128.5,
hide axis
]
\addplot [forget plot] graphics [xmin=0.5,xmax=128.5,ymin=0.5,ymax=128.5] {./images/Q2/black-1.png};
\end{axis}
\end{tikzpicture}%}
			\caption{}
			\label{fig:}
	  	\end{figure}

	\section{Question 3}

		As can be seen in equation \ref{eq:q2:sin_form}, the amplitude of the waveform is
	
		\begin{equation}
			A = \dfrac{1}{N^2}
		\end{equation}
	
	\section{Question 4}
	
		As seen in the lecture notes, 
		
		\begin{equation}
			\lambda = \frac{2 \pi}{|\omega|}
			\label{q4:eq:wavelength}
		\end{equation}
		
		and 
		
		\begin{equation}
			\omega = [\frac{2 \pi u}{N} \ \frac{2 \pi v}{N}]^T
		\end{equation}
		
		Hence, equation \ref{q4:eq:wavelength} for $(u,v)=(p,q)$ becomes
		
		\begin{equation}
			\lambda = \frac{N}{\sqrt{p^2+q^2}}
		\end{equation}
		
		TODO: add figures
		
		
	\section{Question 5}
	
		For an quadratic image of size $N$, the highest number of cycles that can fit in it is $N/2$. 
		Hence, when either $p$ or $q$ exceed the value of $N/2$, which in our case is $N/2 = 64$, the Nyquist frequency is exceeded
		and the corresponding waveform in the spatial domain is no longer a sinusoid.
		
		\begin{figure}[!htb]
			\centering
			\scalebox{.5}{% This file was created by matlab2tikz.
%
%The latest updates can be retrieved from
%  http://www.mathworks.com/matlabcentral/fileexchange/22022-matlab2tikz-matlab2tikz
%where you can also make suggestions and rate matlab2tikz.
%
\definecolor{mycolor1}{rgb}{0.00000,0.44700,0.74100}%
%
\begin{tikzpicture}

\begin{axis}[%
width=10.793in,
height=5.671in,
at={(1.811in,0.765in)},
scale only axis,
xmin=0,
xmax=140,
ymin=-0.002,
ymax=0.002,
axis background/.style={fill=white}
]
\addplot [color=mycolor1,solid,forget plot]
  table[row sep=crcr]{%
1	0.0011761405011441\\
2	-0.00150132552114309\\
3	0.00173652485231055\\
4	-0.00186764123623756\\
5	0.00188681588550005\\
6	-0.00179289951968045\\
7	0.00159152125035495\\
8	-0.00129475118625478\\
9	0.000920376981200602\\
10	-0.000490837686716064\\
11	3.18788115201923e-05\\
12	0.000428990799737703\\
13	-0.000864147777677171\\
14	0.00124750990371305\\
15	-0.00155609941286574\\
16	0.00177142022340174\\
17	-0.00188056654561545\\
18	0.00187699642251639\\
19	-0.00176092383833509\\
20	0.00153930589284421\\
21	-0.00122542581023588\\
22	0.000838096775955735\\
23	-0.0004005343215214\\
24	-6.10351562498684e-05\\
25	0.000518946339733578\\
26	-0.00094575318029509\\
27	0.00131587394565499\\
28	-0.00160712452480459\\
29	0.00180204808801667\\
30	-0.00188896140546677\\
31	0.00186265511074541\\
32	-0.00172470593722446\\
33	0.00148338221261035\\
34	-0.00115314827610533\\
35	0.000753797522168983\\
36	-0.000309266034063528\\
37	-0.000153802084982475\\
38	0.000607651692542704\\
39	-0.00102508018066348\\
40	0.0013810679320049\\
41	-0.00165427793299569\\
42	0.00182833466094679\\
43	-0.00189280559183885\\
44	0.00184382649966333\\
45	-0.00168433306844467\\
46	0.00142388493469737\\
47	-0.00107809270677394\\
48	0.000667682304126462\\
49	-0.000217252697641091\\
50	-0.000246198491020887\\
51	0.000694893159200493\\
52	-0.00110193767309024\\
53	0.00144293480473935\\
54	-0.00169744604074858\\
55	0.0018502166155355\\
56	-0.00189208984375\\
57	0.00182055594904343\\
58	-0.00163990249377283\\
59	0.00136095739325911\\
60	-0.00100043991768217\\
61	0.000579958580931237\\
62	-0.000124715980440949\\
63	-0.000338001783329989\\
64	0.000780460567372316\\
65	-0.00117614050114418\\
66	0.00150132552114302\\
67	-0.00173652485231059\\
68	0.00186764123623754\\
69	-0.00188681588550006\\
70	0.00179289951968042\\
71	-0.00159152125035501\\
72	0.00129475118625471\\
73	-0.000920376981200516\\
74	0.000490837686716177\\
75	-3.1878811520094e-05\\
76	-0.000428990799737589\\
77	0.000864147777677067\\
78	-0.00124750990371312\\
79	0.00155609941286568\\
80	-0.00177142022340178\\
81	0.00188056654561543\\
82	-0.0018769964225164\\
83	0.00176092383833505\\
84	-0.00153930589284428\\
85	0.0012254258102358\\
86	-0.000838096775955647\\
87	0.000400534321521514\\
88	6.10351562499666e-05\\
89	-0.000518946339733466\\
90	0.000945753180294988\\
91	-0.00131587394565506\\
92	0.00160712452480453\\
93	-0.0018020480880167\\
94	0.00188896140546678\\
95	-0.00186265511074543\\
96	0.00172470593722451\\
97	-0.00148338221261029\\
98	0.00115314827610542\\
99	-0.000753797522169091\\
100	0.000309266034063431\\
101	0.000153802084982573\\
102	-0.000607651692542594\\
103	0.00102508018066357\\
104	-0.00138106793200497\\
105	0.00165427793299564\\
106	-0.00182833466094676\\
107	0.00189280559183885\\
108	-0.00184382649966336\\
109	0.00168433306844472\\
110	-0.0014238849346973\\
111	0.00107809270677404\\
112	-0.000667682304126571\\
113	0.000217252697640994\\
114	0.000246198491020984\\
115	-0.000694893159200384\\
116	0.00110193767309032\\
117	-0.00144293480473942\\
118	0.00169744604074853\\
119	-0.00185021661553547\\
120	0.00189208984375\\
121	-0.00182055594904346\\
122	0.00163990249377289\\
123	-0.00136095739325904\\
124	0.00100043991768227\\
125	-0.000579958580931348\\
126	0.00012471598044085\\
127	0.000338001783330086\\
128	-0.000780460567372209\\
};
\end{axis}
\end{tikzpicture}%}
			\caption{Example waveform in the spatial domain for $(p,q)=(69,120)$}
			\label{fig:}
	  	\end{figure}
	  	
	
	\section{Question 6}
	
		The purpose of these lines is to correctly map the angular frequency values $\omega_x$ and $\omega_y$ inside the interval
		
		\begin{equation}
			-\frac{N}{2} \leq \omega_x, \omega_y \leq \frac{N}{2}
		\end{equation}

	

\end{document}