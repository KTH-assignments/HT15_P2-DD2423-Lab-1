\section{Section 1.7 - Rotation}

\subsection{Question 12}

Figure \ref{fig:Q12} illustrates the effect of rotation of the original image to the spectra of the various images. The orientation of each spectrum
follows the rotation of each image, i.e. it is rotated by the same angle and towards the same direction. However, because of the rotation, each
image loses its original smoothness, due to the limited resolution and the nature of the shape of the pixels. This has a direct effect on the Fourier transform
of each image, as is clearly seen by the wave-like patterns in the spectrum of the images rotated by $30$ and $60$ degrees.

\begin{figure}
	\centering
	% This file was created by matlab2tikz.
%
%The latest updates can be retrieved from
%  http://www.mathworks.com/matlabcentral/fileexchange/22022-matlab2tikz-matlab2tikz
%where you can also make suggestions and rate matlab2tikz.
%
\begin{tikzpicture}

\begin{axis}[%
width=0.923in,
height=0.923in,
at={(3.055in,5.943in)},
scale only axis,
axis on top,
separate axis lines,
every outer x axis line/.append style={black},
every x tick label/.append style={font=\color{black}},
xmin=0.5,
xmax=128.5,
every outer y axis line/.append style={black},
every y tick label/.append style={font=\color{black}},
y dir=reverse,
ymin=0.5,
ymax=128.5,
hide axis
]
\addplot [forget plot] graphics [xmin=0.5,xmax=128.5,ymin=0.5,ymax=128.5] {./images/Q12/q12-1.png};
\end{axis}

\begin{axis}[%
width=0.923in,
height=0.923in,
at={(5.226in,5.943in)},
scale only axis,
axis on top,
separate axis lines,
every outer x axis line/.append style={black},
every x tick label/.append style={font=\color{black}},
xmin=0.5,
xmax=128.5,
every outer y axis line/.append style={black},
every y tick label/.append style={font=\color{black}},
y dir=reverse,
ymin=0.5,
ymax=128.5,
hide axis
]
\addplot [forget plot] graphics [xmin=0.5,xmax=128.5,ymin=0.5,ymax=128.5] {./images/Q12/q12-2.png};
\end{axis}

\begin{axis}[%
width=0.923in,
height=0.923in,
at={(3.055in,4.661in)},
scale only axis,
axis on top,
separate axis lines,
every outer x axis line/.append style={black},
every x tick label/.append style={font=\color{black}},
xmin=0.5,
xmax=182.5,
every outer y axis line/.append style={black},
every y tick label/.append style={font=\color{black}},
y dir=reverse,
ymin=0.5,
ymax=182.5,
hide axis
]
\addplot [forget plot] graphics [xmin=0.5,xmax=182.5,ymin=0.5,ymax=182.5] {./images/Q12/q12-3.png};
\end{axis}

\begin{axis}[%
width=0.923in,
height=0.923in,
at={(5.226in,4.661in)},
scale only axis,
axis on top,
separate axis lines,
every outer x axis line/.append style={black},
every x tick label/.append style={font=\color{black}},
xmin=0.5,
xmax=182.5,
every outer y axis line/.append style={black},
every y tick label/.append style={font=\color{black}},
y dir=reverse,
ymin=0.5,
ymax=182.5,
hide axis
]
\addplot [forget plot] graphics [xmin=0.5,xmax=182.5,ymin=0.5,ymax=182.5] {./images/Q12/q12-4.png};
\end{axis}

\begin{axis}[%
width=0.923in,
height=0.923in,
at={(7.398in,4.661in)},
scale only axis,
axis on top,
separate axis lines,
every outer x axis line/.append style={black},
every x tick label/.append style={font=\color{black}},
xmin=0.5,
xmax=258.5,
every outer y axis line/.append style={black},
every y tick label/.append style={font=\color{black}},
y dir=reverse,
ymin=0.5,
ymax=258.5,
hide axis
]
\addplot [forget plot] graphics [xmin=0.5,xmax=258.5,ymin=0.5,ymax=258.5] {./images/Q12/q12-5.png};
\end{axis}

\begin{axis}[%
width=0.923in,
height=0.923in,
at={(3.055in,3.38in)},
scale only axis,
axis on top,
separate axis lines,
every outer x axis line/.append style={black},
every x tick label/.append style={font=\color{black}},
xmin=0.5,
xmax=182.5,
every outer y axis line/.append style={black},
every y tick label/.append style={font=\color{black}},
y dir=reverse,
ymin=0.5,
ymax=182.5,
hide axis
]
\addplot [forget plot] graphics [xmin=0.5,xmax=182.5,ymin=0.5,ymax=182.5] {./images/Q12/q12-6.png};
\end{axis}

\begin{axis}[%
width=0.923in,
height=0.923in,
at={(5.226in,3.38in)},
scale only axis,
axis on top,
separate axis lines,
every outer x axis line/.append style={black},
every x tick label/.append style={font=\color{black}},
xmin=0.5,
xmax=182.5,
every outer y axis line/.append style={black},
every y tick label/.append style={font=\color{black}},
y dir=reverse,
ymin=0.5,
ymax=182.5,
hide axis
]
\addplot [forget plot] graphics [xmin=0.5,xmax=182.5,ymin=0.5,ymax=182.5] {./images/Q12/q12-7.png};
\end{axis}

\begin{axis}[%
width=0.923in,
height=0.923in,
at={(7.398in,3.38in)},
scale only axis,
axis on top,
separate axis lines,
every outer x axis line/.append style={black},
every x tick label/.append style={font=\color{black}},
xmin=0.5,
xmax=258.5,
every outer y axis line/.append style={black},
every y tick label/.append style={font=\color{black}},
y dir=reverse,
ymin=0.5,
ymax=258.5,
hide axis
]
\addplot [forget plot] graphics [xmin=0.5,xmax=258.5,ymin=0.5,ymax=258.5] {./images/Q12/q12-8.png};
\end{axis}

\begin{axis}[%
width=0.923in,
height=0.923in,
at={(3.055in,2.098in)},
scale only axis,
axis on top,
separate axis lines,
every outer x axis line/.append style={black},
every x tick label/.append style={font=\color{black}},
xmin=0.5,
xmax=182.5,
every outer y axis line/.append style={black},
every y tick label/.append style={font=\color{black}},
y dir=reverse,
ymin=0.5,
ymax=182.5,
hide axis
]
\addplot [forget plot] graphics [xmin=0.5,xmax=182.5,ymin=0.5,ymax=182.5] {./images/Q12/q12-9.png};
\end{axis}

\begin{axis}[%
width=0.923in,
height=0.923in,
at={(5.226in,2.098in)},
scale only axis,
axis on top,
separate axis lines,
every outer x axis line/.append style={black},
every x tick label/.append style={font=\color{black}},
xmin=0.5,
xmax=182.5,
every outer y axis line/.append style={black},
every y tick label/.append style={font=\color{black}},
y dir=reverse,
ymin=0.5,
ymax=182.5,
hide axis
]
\addplot [forget plot] graphics [xmin=0.5,xmax=182.5,ymin=0.5,ymax=182.5] {./images/Q12/q12-10.png};
\end{axis}

\begin{axis}[%
width=0.923in,
height=0.923in,
at={(7.398in,2.098in)},
scale only axis,
axis on top,
separate axis lines,
every outer x axis line/.append style={black},
every x tick label/.append style={font=\color{black}},
xmin=0.5,
xmax=258.5,
every outer y axis line/.append style={black},
every y tick label/.append style={font=\color{black}},
y dir=reverse,
ymin=0.5,
ymax=258.5,
hide axis
]
\addplot [forget plot] graphics [xmin=0.5,xmax=258.5,ymin=0.5,ymax=258.5] {./images/Q12/q12-11.png};
\end{axis}

\begin{axis}[%
width=0.923in,
height=0.923in,
at={(3.055in,0.816in)},
scale only axis,
axis on top,
separate axis lines,
every outer x axis line/.append style={black},
every x tick label/.append style={font=\color{black}},
xmin=0.5,
xmax=182.5,
every outer y axis line/.append style={black},
every y tick label/.append style={font=\color{black}},
y dir=reverse,
ymin=0.5,
ymax=182.5,
hide axis
]
\addplot [forget plot] graphics [xmin=0.5,xmax=182.5,ymin=0.5,ymax=182.5] {./images/Q12/q12-12.png};
\end{axis}

\begin{axis}[%
width=0.923in,
height=0.923in,
at={(5.226in,0.816in)},
scale only axis,
axis on top,
separate axis lines,
every outer x axis line/.append style={black},
every x tick label/.append style={font=\color{black}},
xmin=0.5,
xmax=182.5,
every outer y axis line/.append style={black},
every y tick label/.append style={font=\color{black}},
y dir=reverse,
ymin=0.5,
ymax=182.5,
hide axis
]
\addplot [forget plot] graphics [xmin=0.5,xmax=182.5,ymin=0.5,ymax=182.5] {./images/Q12/q12-13.png};
\end{axis}

\begin{axis}[%
width=0.923in,
height=0.923in,
at={(7.398in,0.816in)},
scale only axis,
axis on top,
separate axis lines,
every outer x axis line/.append style={black},
every x tick label/.append style={font=\color{black}},
xmin=0.5,
xmax=258.5,
every outer y axis line/.append style={black},
every y tick label/.append style={font=\color{black}},
y dir=reverse,
ymin=0.5,
ymax=258.5,
hide axis
]
\addplot [forget plot] graphics [xmin=0.5,xmax=258.5,ymin=0.5,ymax=258.5] {./images/Q12/q12-14.png};
\end{axis}
\end{tikzpicture}%
	\caption{The first row depicts the original $F$ image and its Fourier transform. The next images in the first column represent $F$ rotated by 
	a) $30$, b) $45$, c) $60$ and d) $90$ degrees respectively. 
	The second column features their respective Fourier transforms. The third column features the Fourier spectra of the second column
	rotated, so as to match the orientation of the Fourier spectrum of image $F$.}
	\label{fig:Q12}
\end{figure}

The rotation of the image can be seen as the rotation of a sum of pixels. If a pixel with coordinates $A(x,y)$ with respect
to the center of the image is rotated by an angle $\theta$, then its new coordinates can be expressed as 
$$A(x',y') = A(x cos\theta + ysin\theta, -x sin\theta + y cos\theta)$$
Hence, $$x = x' cos\theta - y' sin\theta$$ and $$y = x' sin\theta + y' cos\theta$$

If we now take the Fourier transform of the rotated image $f(x',y')$, then

\begin{multline}
\mathcal{F}(f') = \sum_{x=0}^{N-1} \sum_{y=0}^{N-1} f(x',y') \cdot e^{-\dfrac{2 \pi i (xu + yv)}{N}} = \\
\sum_{x=0}^{N-1} \sum_{y=0}^{N-1} f(x',y') \cdot e^{-\dfrac{2 \pi i ((x' cos\theta - y' sin\theta)u + (x' sin\theta + y' cos\theta)v)}{N}} = \\
\sum_{x=0}^{N-1} \sum_{y=0}^{N-1} f(x',y') \cdot e^{-\dfrac{2 \pi i (x' (u cos\theta + v sin\theta) + y' (v cos\theta -u sin\theta))}{N}} = \\
\sum_{x'=0}^{N-1} \sum_{y'=0}^{N-1} f(x',y') \cdot e^{-\dfrac{2 \pi i (x' (u cos\theta + v sin\theta) + y' (v cos\theta -u sin\theta))}{N}}
\end{multline}

which means that the rotation is propagated to the frequency domain:

$$u' = u cos\theta + v sin\theta$$ and $$v' = v cos\theta -u sin\theta$$