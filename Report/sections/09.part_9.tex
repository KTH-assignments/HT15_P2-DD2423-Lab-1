\section{Section 3.2 - Smoothing and subsampling}

Figures \ref{fig:Q19_ideal_05}, \ref{fig:Q19_ideal_06}, \ref{fig:Q19_gauss_1} and \ref{fig:Q19_gauss_1} 
illustrate image \texttt{hand256} subsampled and smoothed using a Gaussian and an ideal low-pass filter.


    \begin{figure}[H]
    	\centering
		\scalebox{1}{\input{./images/Q19/ideal_05.tex}}
      	\caption{Image \texttt{hand256} subsampled $1,2,3$ and $4$ times are presented at row $1$ respectively. 
      	Row $2$ illustrates their respective Fourier transforms. Row $3$ features the smoothed version of the images in row $1$ using the
      	\texttt{ideal} operator with $CUTOFF=0.5$. Row $4$ illustrates the Fourier transform of the smoothed images.}
      	\label{fig:Q19_ideal_05}
    \end{figure}
    
    \begin{figure}[H]
    	\centering
		\scalebox{1}{\input{./images/Q19/ideal_06.tex}}
      	\caption{Image \texttt{hand256} subsampled $1,2,3$ and $4$ times are presented at row $1$ respectively. 
      	Row $2$ illustrates their respective Fourier transforms. Row $3$ features the smoothed version of the images in row $1$ using the
      	\texttt{ideal} operator with $CUTOFF=0.6$. Row $4$ illustrates the Fourier transform of the smoothed images.}
      	\label{fig:Q19_ideal_06}
    \end{figure}
    
    
    \begin{figure}[H]
    	\centering
		\scalebox{1}{\input{./images/Q19/gauss_1.tex}}
      	\caption{Image \texttt{hand256} subsampled $1,2,3$ and $4$ times are presented at row $1$ respectively. 
      	Row $2$ illustrates their respective Fourier transforms. Row $3$ features the smoothed version of the images in row $1$ using the
      	\texttt{gaussfft} operator with $t=1$. Row $4$ illustrates the Fourier transform of the smoothed images.}
      	\label{fig:Q19_gauss_1}
    \end{figure}
    
    \begin{figure}[H]
    	\centering
		\scalebox{1}{% This file was created by matlab2tikz.
%
%The latest updates can be retrieved from
%  http://www.mathworks.com/matlabcentral/fileexchange/22022-matlab2tikz-matlab2tikz
%where you can also make suggestions and rate matlab2tikz.
%
\begin{tikzpicture}

\begin{axis}[%
width=1.171in,
height=1.171in,
at={(2.5in,5.695in)},
scale only axis,
axis on top,
separate axis lines,
every outer x axis line/.append style={black},
every x tick label/.append style={font=\color{black}},
xmin=0.5,
xmax=256.5,
every outer y axis line/.append style={black},
every y tick label/.append style={font=\color{black}},
y dir=reverse,
ymin=0.5,
ymax=256.5,
hide axis
]
\addplot [forget plot] graphics [xmin=0.5,xmax=256.5,ymin=0.5,ymax=256.5] {./images/Q19/gauss_07-1.png};
\end{axis}

\begin{axis}[%
width=1.171in,
height=1.171in,
at={(2.5in,4.069in)},
scale only axis,
axis on top,
separate axis lines,
every outer x axis line/.append style={black},
every x tick label/.append style={font=\color{black}},
xmin=0.5,
xmax=256.5,
every outer y axis line/.append style={black},
every y tick label/.append style={font=\color{black}},
y dir=reverse,
ymin=0.5,
ymax=256.5,
hide axis
]
\addplot [forget plot] graphics [xmin=0.5,xmax=256.5,ymin=0.5,ymax=256.5] {./images/Q19/gauss_07-2.png};
\end{axis}

\begin{axis}[%
width=1.171in,
height=1.171in,
at={(2.5in,2.443in)},
scale only axis,
axis on top,
separate axis lines,
every outer x axis line/.append style={black},
every x tick label/.append style={font=\color{black}},
xmin=0.5,
xmax=256.5,
every outer y axis line/.append style={black},
every y tick label/.append style={font=\color{black}},
y dir=reverse,
ymin=0.5,
ymax=256.5,
hide axis
]
\addplot [forget plot] graphics [xmin=0.5,xmax=256.5,ymin=0.5,ymax=256.5] {./images/Q19/gauss_07-3.png};
\end{axis}

\begin{axis}[%
width=1.171in,
height=1.171in,
at={(2.5in,0.816in)},
scale only axis,
axis on top,
separate axis lines,
every outer x axis line/.append style={black},
every x tick label/.append style={font=\color{black}},
xmin=0.5,
xmax=256.5,
every outer y axis line/.append style={black},
every y tick label/.append style={font=\color{black}},
y dir=reverse,
ymin=0.5,
ymax=256.5,
hide axis
]
\addplot [forget plot] graphics [xmin=0.5,xmax=256.5,ymin=0.5,ymax=256.5] {./images/Q19/gauss_07-4.png};
\end{axis}

\begin{axis}[%
width=1.171in,
height=1.171in,
at={(4.5in,5.695in)},
scale only axis,
axis on top,
separate axis lines,
every outer x axis line/.append style={black},
every x tick label/.append style={font=\color{black}},
xmin=0.5,
xmax=128.5,
every outer y axis line/.append style={black},
every y tick label/.append style={font=\color{black}},
y dir=reverse,
ymin=0.5,
ymax=128.5,
hide axis
]
\addplot [forget plot] graphics [xmin=0.5,xmax=128.5,ymin=0.5,ymax=128.5] {./images/Q19/gauss_07-5.png};
\end{axis}

\begin{axis}[%
width=1.171in,
height=1.171in,
at={(4.5in,4.069in)},
scale only axis,
axis on top,
separate axis lines,
every outer x axis line/.append style={black},
every x tick label/.append style={font=\color{black}},
xmin=0.5,
xmax=128.5,
every outer y axis line/.append style={black},
every y tick label/.append style={font=\color{black}},
y dir=reverse,
ymin=0.5,
ymax=128.5,
hide axis
]
\addplot [forget plot] graphics [xmin=0.5,xmax=128.5,ymin=0.5,ymax=128.5] {./images/Q19/gauss_07-6.png};
\end{axis}

\begin{axis}[%
width=1.171in,
height=1.171in,
at={(4.5in,2.443in)},
scale only axis,
axis on top,
separate axis lines,
every outer x axis line/.append style={black},
every x tick label/.append style={font=\color{black}},
xmin=0.5,
xmax=128.5,
every outer y axis line/.append style={black},
every y tick label/.append style={font=\color{black}},
y dir=reverse,
ymin=0.5,
ymax=128.5,
hide axis
]
\addplot [forget plot] graphics [xmin=0.5,xmax=128.5,ymin=0.5,ymax=128.5] {./images/Q19/gauss_07-7.png};
\end{axis}

\begin{axis}[%
width=1.171in,
height=1.171in,
at={(4.5in,0.816in)},
scale only axis,
axis on top,
separate axis lines,
every outer x axis line/.append style={black},
every x tick label/.append style={font=\color{black}},
xmin=0.5,
xmax=128.5,
every outer y axis line/.append style={black},
every y tick label/.append style={font=\color{black}},
y dir=reverse,
ymin=0.5,
ymax=128.5,
hide axis
]
\addplot [forget plot] graphics [xmin=0.5,xmax=128.5,ymin=0.5,ymax=128.5] {./images/Q19/gauss_07-8.png};
\end{axis}

\begin{axis}[%
width=1.171in,
height=1.171in,
at={(6.5in,5.695in)},
scale only axis,
axis on top,
separate axis lines,
every outer x axis line/.append style={black},
every x tick label/.append style={font=\color{black}},
xmin=0.5,
xmax=64.5,
every outer y axis line/.append style={black},
every y tick label/.append style={font=\color{black}},
y dir=reverse,
ymin=0.5,
ymax=64.5,
hide axis
]
\addplot [forget plot] graphics [xmin=0.5,xmax=64.5,ymin=0.5,ymax=64.5] {./images/Q19/gauss_07-9.png};
\end{axis}

\begin{axis}[%
width=1.171in,
height=1.171in,
at={(6.5in,4.069in)},
scale only axis,
axis on top,
separate axis lines,
every outer x axis line/.append style={black},
every x tick label/.append style={font=\color{black}},
xmin=0.5,
xmax=64.5,
every outer y axis line/.append style={black},
every y tick label/.append style={font=\color{black}},
y dir=reverse,
ymin=0.5,
ymax=64.5,
hide axis
]
\addplot [forget plot] graphics [xmin=0.5,xmax=64.5,ymin=0.5,ymax=64.5] {./images/Q19/gauss_07-10.png};
\end{axis}

\begin{axis}[%
width=1.171in,
height=1.171in,
at={(6.5in,2.443in)},
scale only axis,
axis on top,
separate axis lines,
every outer x axis line/.append style={black},
every x tick label/.append style={font=\color{black}},
xmin=0.5,
xmax=64.5,
every outer y axis line/.append style={black},
every y tick label/.append style={font=\color{black}},
y dir=reverse,
ymin=0.5,
ymax=64.5,
hide axis
]
\addplot [forget plot] graphics [xmin=0.5,xmax=64.5,ymin=0.5,ymax=64.5] {./images/Q19/gauss_07-11.png};
\end{axis}

\begin{axis}[%
width=1.171in,
height=1.171in,
at={(6.5in,0.816in)},
scale only axis,
axis on top,
separate axis lines,
every outer x axis line/.append style={black},
every x tick label/.append style={font=\color{black}},
xmin=0.5,
xmax=64.5,
every outer y axis line/.append style={black},
every y tick label/.append style={font=\color{black}},
y dir=reverse,
ymin=0.5,
ymax=64.5,
hide axis
]
\addplot [forget plot] graphics [xmin=0.5,xmax=64.5,ymin=0.5,ymax=64.5] {./images/Q19/gauss_07-12.png};
\end{axis}

\begin{axis}[%
width=1.171in,
height=1.171in,
at={(8.5in,5.695in)},
scale only axis,
axis on top,
separate axis lines,
every outer x axis line/.append style={black},
every x tick label/.append style={font=\color{black}},
xmin=0.5,
xmax=32.5,
every outer y axis line/.append style={black},
every y tick label/.append style={font=\color{black}},
y dir=reverse,
ymin=0.5,
ymax=32.5,
hide axis
]
\addplot [forget plot] graphics [xmin=0.5,xmax=32.5,ymin=0.5,ymax=32.5] {./images/Q19/gauss_07-13.png};
\end{axis}

\begin{axis}[%
width=1.171in,
height=1.171in,
at={(8.5in,4.069in)},
scale only axis,
axis on top,
separate axis lines,
every outer x axis line/.append style={black},
every x tick label/.append style={font=\color{black}},
xmin=0.5,
xmax=32.5,
every outer y axis line/.append style={black},
every y tick label/.append style={font=\color{black}},
y dir=reverse,
ymin=0.5,
ymax=32.5,
hide axis
]
\addplot [forget plot] graphics [xmin=0.5,xmax=32.5,ymin=0.5,ymax=32.5] {./images/Q19/gauss_07-14.png};
\end{axis}

\begin{axis}[%
width=1.171in,
height=1.171in,
at={(8.5in,2.443in)},
scale only axis,
axis on top,
separate axis lines,
every outer x axis line/.append style={black},
every x tick label/.append style={font=\color{black}},
xmin=0.5,
xmax=32.5,
every outer y axis line/.append style={black},
every y tick label/.append style={font=\color{black}},
y dir=reverse,
ymin=0.5,
ymax=32.5,
hide axis
]
\addplot [forget plot] graphics [xmin=0.5,xmax=32.5,ymin=0.5,ymax=32.5] {./images/Q19/gauss_07-15.png};
\end{axis}

\begin{axis}[%
width=1.171in,
height=1.171in,
at={(8.5in,0.816in)},
scale only axis,
axis on top,
separate axis lines,
every outer x axis line/.append style={black},
every x tick label/.append style={font=\color{black}},
xmin=0.5,
xmax=32.5,
every outer y axis line/.append style={black},
every y tick label/.append style={font=\color{black}},
y dir=reverse,
ymin=0.5,
ymax=32.5,
hide axis
]
\addplot [forget plot] graphics [xmin=0.5,xmax=32.5,ymin=0.5,ymax=32.5] {./images/Q19/gauss_07-16.png};
\end{axis}
\end{tikzpicture}%}
      	\caption{Image \texttt{hand256} subsampled $1,2,3$ and $4$ times are presented at row $1$ respectively. 
      	Row $2$ illustrates their respective Fourier transforms. Row $3$ features the smoothed version of the images in row $1$ using the
      	\texttt{gaussfft} operator with $t=0.7$. Row $4$ illustrates the Fourier transform of the smoothed images.}
      	\label{fig:Q19_gauss_07}
    \end{figure}
    
    
\subsection{Question 19}

Sub-sampling results in pixels of bigger size, hence a neighbourhood of pixels are compacted into one, and the different values of all those pixels are lost.
Hence, information is lost and shapes become coarser. Is the sub-sampling occurs at a lower frequency than the Nyquist frequency then the image's 
characteristics are distorted irrevocably.

The first thing noticed in this exercise is that there is an qualitative information loss balance: it is possible to smooth the sub-sampled images to a higher degree 
than the original images. Smoothing the original images in that degree results in information loss, just as the one introduced when sub-sampling. Since the two
filters used here are low-pass filters, we can see that the outline of the hand in the image remains fairly accurate, even a higher variances, or lower cut-off
frequencies. 


    \begin{figure}[H]
    	\centering
		\scalebox{1}{\input{./images/Q19/q19_N4_gauss_5.tex}}
      	\caption{The upper left figure shows image \texttt{hand256} subsampled $4$ times. The lower left figure shows its smoothed version
      	using a Gaussian filter with $t=5$. The figures on the right-hand side illustrate their corresponding spectra.}
      	\label{fig:Q19_N4_gauss_5}
    \end{figure}
    
     \begin{figure}[H]
    	\centering
		\scalebox{1}{\input{./images/Q19/q19_N1_gauss_5.tex}}
      	\caption{The uppermost figure shows image \texttt{hand256}. The left figure shows its smoothed version
      	using a Gaussian filter with $t=100$. The figures on the right-hand side illustrate their corresponding spectra. Figure used for comparison with
      	the above figure.}
      	\label{fig:Q19_N4_gauss_100}
    \end{figure}
    
    \begin{figure}[H]
    	\centering
		\scalebox{1}{% This file was created by matlab2tikz.
%
%The latest updates can be retrieved from
%  http://www.mathworks.com/matlabcentral/fileexchange/22022-matlab2tikz-matlab2tikz
%where you can also make suggestions and rate matlab2tikz.
%
\begin{tikzpicture}

\begin{axis}[%
width=1.456in,
height=1.456in,
at={(0.964in,2.491in)},
scale only axis,
axis on top,
separate axis lines,
every outer x axis line/.append style={black},
every x tick label/.append style={font=\color{black}},
xmin=0.5,
xmax=32.5,
every outer y axis line/.append style={black},
every y tick label/.append style={font=\color{black}},
y dir=reverse,
ymin=0.5,
ymax=32.5,
hide axis
]
\addplot [forget plot] graphics [xmin=0.5,xmax=32.5,ymin=0.5,ymax=32.5] {./images/Q19/q19_N4_ideal_01-1.png};
\end{axis}

\begin{axis}[%
width=1.456in,
height=1.456in,
at={(3.469in,2.491in)},
scale only axis,
axis on top,
separate axis lines,
every outer x axis line/.append style={black},
every x tick label/.append style={font=\color{black}},
xmin=0.5,
xmax=32.5,
every outer y axis line/.append style={black},
every y tick label/.append style={font=\color{black}},
y dir=reverse,
ymin=0.5,
ymax=32.5,
hide axis
]
\addplot [forget plot] graphics [xmin=0.5,xmax=32.5,ymin=0.5,ymax=32.5] {./images/Q19/q19_N4_ideal_01-2.png};
\end{axis}

\begin{axis}[%
width=1.456in,
height=1.456in,
at={(0.964in,0.469in)},
scale only axis,
axis on top,
separate axis lines,
every outer x axis line/.append style={black},
every x tick label/.append style={font=\color{black}},
xmin=0.5,
xmax=32.5,
every outer y axis line/.append style={black},
every y tick label/.append style={font=\color{black}},
y dir=reverse,
ymin=0.5,
ymax=32.5,
hide axis
]
\addplot [forget plot] graphics [xmin=0.5,xmax=32.5,ymin=0.5,ymax=32.5] {./images/Q19/q19_N4_ideal_01-3.png};
\end{axis}

\begin{axis}[%
width=1.456in,
height=1.456in,
at={(3.469in,0.469in)},
scale only axis,
axis on top,
separate axis lines,
every outer x axis line/.append style={black},
every x tick label/.append style={font=\color{black}},
xmin=0.5,
xmax=32.5,
every outer y axis line/.append style={black},
every y tick label/.append style={font=\color{black}},
y dir=reverse,
ymin=0.5,
ymax=32.5,
hide axis
]
\addplot [forget plot] graphics [xmin=0.5,xmax=32.5,ymin=0.5,ymax=32.5] {./images/Q19/q19_N4_ideal_01-4.png};
\end{axis}
\end{tikzpicture}%}
      	\caption{The upper left figure shows image \texttt{hand256} subsampled $4$ times. The lower left figure shows its smoothed version
      	using an ideal low-pass filter with $CUTOFF=0.1$. The figures on the right-hand side illustrate their corresponding spectra.}
      	\label{fig:Q19_N4_ideal_01}
    \end{figure}
    
    \begin{figure}[H]
    	\centering
		\scalebox{1}{\input{./images/Q19/q19_N1_ideal_01.tex}}
      	\caption{The upper left figure shows image \texttt{hand256} subsampled $4$ times. The lower left figure shows its smoothed version
      	using an ideal low-pass filter with $CUTOFF=0.05$. The figures on the right-hand side illustrate their corresponding spectra. Figure used for comparison with
      	the above figure.}
      	\label{fig:Q19_N4_ideal_005}
    \end{figure}
    
