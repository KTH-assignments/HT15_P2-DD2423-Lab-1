\section{Section 1.6 - Scaling}

\subsection{Question 11}

Image $F$ and its Fourier transform are illustrated in figures \ref{fig:Q11_1} and \ref{fig:Q11_2} respectively.

\begin{minipage}{\linewidth}
  \centering
  \begin{minipage}{0.4\linewidth}
    \begin{figure}[H]
      \scalebox{0.6}{\input{./images/Q11/q11_1.tex}}
      \caption{Image $F$}
      \label{fig:Q11_1}
    \end{figure}
  \end{minipage}
  \hspace{0.05\linewidth}
  \begin{minipage}{0.4\linewidth}
    \begin{figure}[H]
      \scalebox{0.6}{\input{./images/Q11/q11_2.tex}}
      \caption{Image $\mathcal{F}(F)$.}
      \label{fig:Q11_2}
    \end{figure}
  \end{minipage}
\end{minipage}
\\

Compared to figure \ref{fig:Q10_1}, the height of the non-zero area in image \ref{fig:Q11_1} is cut in half, while its width is increased by a factor of $2$.
Comparing their respective Fourier transforms verifies the transform's scaling property: a compression in either the spatial or the frequency domain is expressed
as an expansion in the other.

In formal manner, if $f_1(x,y)$ is the function represented in figure \ref{fig:Q10_1} and $f_2(x,y)$ the function represented in figure \ref{fig:Q11_1},
then $$f_2(x,y) = f_1(\frac{x}{2}, 2y)$$ hence due to the scaling property of the Fourier transform $$f(ax,by) \Longleftrightarrow \dfrac{1}{|ab|}F(\frac{u}{a}, \frac{v}{b})$$ $$F_2(u,v) = F_1(2u, \frac{v}{2})$$ which is why the number of spectrum zeros appear to be twice as many horizontally and only half vertically.