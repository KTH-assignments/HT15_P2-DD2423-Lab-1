\section{Section 1.4 - Linearity}

Figures \ref{fig:02.F}, \ref{fig:02.G} and \ref{fig:02.H} show the spatial $F$, $G$ and $H$ images.
Figures \ref{fig:02.Fhat_nc}, \ref{fig:02.Ghat_nc} and \ref{fig:02.Hhat_nc} show their respective Fourier transforms, with the origin
being in the upper left corner. Figures \ref{fig:02.Fhat_c}, \ref{fig:02.Ghat_c} and \ref{fig:02.Hhat_c} show the Fourier transforms of images $F$, $G$ and $H$
with the origin in the middle of the image.

Here, the \texttt{fftshift()} method is used to shift the origin $O(0,0)$ from the upper left corner to the middle of each image.
The \texttt{log} command is used to make details in the images in the frequency domain visible. 
Figures \ref{fig:02.Fhat_c_nl}, \ref{fig:02.Ghat_c_nl} and \ref{fig:02.Hhat_c_nl} illustrate the Fourier transforms of images $F$, $G$ and $H$
with the origin in the middle of the image without the use of the \texttt{log} command.

\begin{minipage}{\linewidth}
  \centering
  \begin{minipage}{0.25\linewidth}
    \vspace{-0.15\linewidth}
    \begin{figure}[H]
      \scalebox{0.4}{% This file was created by matlab2tikz.
%
%The latest updates can be retrieved from
%  http://www.mathworks.com/matlabcentral/fileexchange/22022-matlab2tikz-matlab2tikz
%where you can also make suggestions and rate matlab2tikz.
%
\begin{tikzpicture}

\begin{axis}[%
width=3.477in,
height=3.477in,
at={(1.205in,0.469in)},
scale only axis,
axis on top,
separate axis lines,
every outer x axis line/.append style={black},
every x tick label/.append style={font=\color{black}},
xmin=0.5,
xmax=128.5,
every outer y axis line/.append style={black},
every y tick label/.append style={font=\color{black}},
y dir=reverse,
ymin=0.5,
ymax=128.5,
hide axis
]
\addplot [forget plot] graphics [xmin=0.5,xmax=128.5,ymin=0.5,ymax=128.5] {./images/Q6/F-1.png};
\end{axis}
\end{tikzpicture}%}
      \caption{Image \texttt{F}}
      \label{fig:02.F}
    \end{figure}
  \end{minipage}
  \hspace{0.05\linewidth}
  \begin{minipage}{0.25\linewidth}
    \begin{figure}[H]
      \scalebox{0.4}{% This file was created by matlab2tikz.
%
%The latest updates can be retrieved from
%  http://www.mathworks.com/matlabcentral/fileexchange/22022-matlab2tikz-matlab2tikz
%where you can also make suggestions and rate matlab2tikz.
%
\begin{tikzpicture}

\begin{axis}[%
width=3.477in,
height=3.477in,
at={(1.205in,0.469in)},
scale only axis,
axis on top,
separate axis lines,
every outer x axis line/.append style={black},
every x tick label/.append style={font=\color{black}},
xmin=0.5,
xmax=128.5,
every outer y axis line/.append style={black},
every y tick label/.append style={font=\color{black}},
y dir=reverse,
ymin=0.5,
ymax=128.5,
hide axis
]
\addplot [forget plot] graphics [xmin=0.5,xmax=128.5,ymin=0.5,ymax=128.5] {./images/Q6/G-1.png};
\end{axis}
\end{tikzpicture}%}
      \caption{Image \texttt{G = F'}}
      \label{fig:02.G}
    \end{figure}
  \end{minipage}
  \hspace{0.05\linewidth}
  \begin{minipage}{0.25\linewidth}
    \begin{figure}[H]
      \scalebox{0.4}{% This file was created by matlab2tikz.
%
%The latest updates can be retrieved from
%  http://www.mathworks.com/matlabcentral/fileexchange/22022-matlab2tikz-matlab2tikz
%where you can also make suggestions and rate matlab2tikz.
%
\begin{tikzpicture}

\begin{axis}[%
width=3.477in,
height=3.477in,
at={(1.205in,0.469in)},
scale only axis,
axis on top,
separate axis lines,
every outer x axis line/.append style={black},
every x tick label/.append style={font=\color{black}},
xmin=0.5,
xmax=128.5,
every outer y axis line/.append style={black},
every y tick label/.append style={font=\color{black}},
y dir=reverse,
ymin=0.5,
ymax=128.5,
hide axis
]
\addplot [forget plot] graphics [xmin=0.5,xmax=128.5,ymin=0.5,ymax=128.5] {./images/Q6/H-1.png};
\end{axis}
\end{tikzpicture}%}
      \caption{Image \texttt{H = F + 2 * G}}
      \label{fig:02.H}
    \end{figure}
  \end{minipage}
\end{minipage}


\begin{minipage}{\linewidth}
  \centering
  \begin{minipage}{0.25\linewidth}
    \begin{figure}[H]
      \scalebox{0.23}{% This file was created by matlab2tikz.
%
%The latest updates can be retrieved from
%  http://www.mathworks.com/matlabcentral/fileexchange/22022-matlab2tikz-matlab2tikz
%where you can also make suggestions and rate matlab2tikz.
%
\begin{tikzpicture}

\begin{axis}[%
width=6.049in,
height=6.049in,
at={(4.663in,0.816in)},
scale only axis,
axis on top,
separate axis lines,
every outer x axis line/.append style={black},
every x tick label/.append style={font=\color{black}},
xmin=0.5,
xmax=128.5,
every outer y axis line/.append style={black},
every y tick label/.append style={font=\color{black}},
y dir=reverse,
ymin=0.5,
ymax=128.5,
hide axis
]
\addplot [forget plot] graphics [xmin=0.5,xmax=128.5,ymin=0.5,ymax=128.5] {./images/Q6/Fhat_nc-1.png};
\end{axis}
\end{tikzpicture}%}
      \caption{Image $\mathcal{F}(F)$. Origin at the upper left corner.}
      \label{fig:02.Fhat_nc}
    \end{figure}
  \end{minipage}
  \hspace{0.05\linewidth}
  \begin{minipage}{0.25\linewidth}
    \begin{figure}[H]
      \scalebox{0.4}{% This file was created by matlab2tikz.
%
%The latest updates can be retrieved from
%  http://www.mathworks.com/matlabcentral/fileexchange/22022-matlab2tikz-matlab2tikz
%where you can also make suggestions and rate matlab2tikz.
%
\begin{tikzpicture}

\begin{axis}[%
width=3.477in,
height=3.477in,
at={(1.205in,0.469in)},
scale only axis,
axis on top,
separate axis lines,
every outer x axis line/.append style={black},
every x tick label/.append style={font=\color{black}},
xmin=0.5,
xmax=128.5,
every outer y axis line/.append style={black},
every y tick label/.append style={font=\color{black}},
y dir=reverse,
ymin=0.5,
ymax=128.5,
hide axis
]
\addplot [forget plot] graphics [xmin=0.5,xmax=128.5,ymin=0.5,ymax=128.5] {./images/Q6/Ghat_nc-1.png};
\end{axis}
\end{tikzpicture}%}
      \caption{Image $\mathcal{F}(G)$. Origin at the upper left corner.}
      \label{fig:02.Ghat_nc}
    \end{figure}
  \end{minipage}
  \hspace{0.05\linewidth}
  \begin{minipage}{0.25\linewidth}
    \begin{figure}[H]
      \scalebox{0.4}{% This file was created by matlab2tikz.
%
%The latest updates can be retrieved from
%  http://www.mathworks.com/matlabcentral/fileexchange/22022-matlab2tikz-matlab2tikz
%where you can also make suggestions and rate matlab2tikz.
%
\begin{tikzpicture}

\begin{axis}[%
width=3.477in,
height=3.477in,
at={(1.205in,0.469in)},
scale only axis,
axis on top,
separate axis lines,
every outer x axis line/.append style={black},
every x tick label/.append style={font=\color{black}},
xmin=0.5,
xmax=128.5,
every outer y axis line/.append style={black},
every y tick label/.append style={font=\color{black}},
y dir=reverse,
ymin=0.5,
ymax=128.5,
hide axis
]
\addplot [forget plot] graphics [xmin=0.5,xmax=128.5,ymin=0.5,ymax=128.5] {./images/Q6/Hhat_nc-1.png};
\end{axis}
\end{tikzpicture}%}
      \caption{Image $\mathcal{F}(H)$. Origin at the upper left corner.}
      \label{fig:02.Hhat_nc}
    \end{figure}
  \end{minipage}
\end{minipage}


\begin{minipage}{\linewidth}
  \centering
  \begin{minipage}{0.25\linewidth}
    \begin{figure}[H]
      \scalebox{0.4}{% This file was created by matlab2tikz.
%
%The latest updates can be retrieved from
%  http://www.mathworks.com/matlabcentral/fileexchange/22022-matlab2tikz-matlab2tikz
%where you can also make suggestions and rate matlab2tikz.
%
\begin{tikzpicture}

\begin{axis}[%
width=3.477in,
height=3.477in,
at={(1.205in,0.469in)},
scale only axis,
axis on top,
separate axis lines,
every outer x axis line/.append style={black},
every x tick label/.append style={font=\color{black}},
xmin=0.5,
xmax=128.5,
every outer y axis line/.append style={black},
every y tick label/.append style={font=\color{black}},
y dir=reverse,
ymin=0.5,
ymax=128.5,
hide axis
]
\addplot [forget plot] graphics [xmin=0.5,xmax=128.5,ymin=0.5,ymax=128.5] {./images/Q6/Fhat_c-1.png};
\end{axis}
\end{tikzpicture}%}
      \caption{Image $\mathcal{F}(F)$. Origin at the middle.}
      \label{fig:02.Fhat_c}
    \end{figure}
  \end{minipage}
  \hspace{0.05\linewidth}
  \begin{minipage}{0.25\linewidth}
    \begin{figure}[H]
      \scalebox{0.4}{% This file was created by matlab2tikz.
%
%The latest updates can be retrieved from
%  http://www.mathworks.com/matlabcentral/fileexchange/22022-matlab2tikz-matlab2tikz
%where you can also make suggestions and rate matlab2tikz.
%
\begin{tikzpicture}

\begin{axis}[%
width=3.477in,
height=3.477in,
at={(1.205in,0.469in)},
scale only axis,
axis on top,
separate axis lines,
every outer x axis line/.append style={black},
every x tick label/.append style={font=\color{black}},
xmin=0.5,
xmax=128.5,
every outer y axis line/.append style={black},
every y tick label/.append style={font=\color{black}},
y dir=reverse,
ymin=0.5,
ymax=128.5,
hide axis
]
\addplot [forget plot] graphics [xmin=0.5,xmax=128.5,ymin=0.5,ymax=128.5] {./images/Q6/Ghat_c-1.png};
\end{axis}
\end{tikzpicture}%}
      \caption{Image $\mathcal{F}(G)$. Origin at the middle.}
      \label{fig:02.Ghat_c}
    \end{figure}
  \end{minipage}
  \hspace{0.05\linewidth}
  \begin{minipage}{0.25\linewidth}
    \begin{figure}[H]
      \scalebox{0.4}{% This file was created by matlab2tikz.
%
%The latest updates can be retrieved from
%  http://www.mathworks.com/matlabcentral/fileexchange/22022-matlab2tikz-matlab2tikz
%where you can also make suggestions and rate matlab2tikz.
%
\begin{tikzpicture}

\begin{axis}[%
width=3.477in,
height=3.477in,
at={(1.205in,0.469in)},
scale only axis,
axis on top,
separate axis lines,
every outer x axis line/.append style={black},
every x tick label/.append style={font=\color{black}},
xmin=0.5,
xmax=128.5,
every outer y axis line/.append style={black},
every y tick label/.append style={font=\color{black}},
y dir=reverse,
ymin=0.5,
ymax=128.5,
hide axis
]
\addplot [forget plot] graphics [xmin=0.5,xmax=128.5,ymin=0.5,ymax=128.5] {./images/Q6/Hhat_c-1.png};
\end{axis}
\end{tikzpicture}%}
      \caption{Image $\mathcal{F}(H)$. Origin at the middle.}
      \label{fig:02.Hhat_c}
    \end{figure}
  \end{minipage}
\end{minipage}


\begin{minipage}{\linewidth}
  \centering
  \begin{minipage}{0.25\linewidth}
    \begin{figure}[H]
      \scalebox{0.4}{% This file was created by matlab2tikz.
%
%The latest updates can be retrieved from
%  http://www.mathworks.com/matlabcentral/fileexchange/22022-matlab2tikz-matlab2tikz
%where you can also make suggestions and rate matlab2tikz.
%
\begin{tikzpicture}

\begin{axis}[%
width=3.477in,
height=3.477in,
at={(1.205in,0.469in)},
scale only axis,
axis on top,
separate axis lines,
every outer x axis line/.append style={black},
every x tick label/.append style={font=\color{black}},
xmin=0.5,
xmax=128.5,
every outer y axis line/.append style={black},
every y tick label/.append style={font=\color{black}},
y dir=reverse,
ymin=0.5,
ymax=128.5,
hide axis
]
\addplot [forget plot] graphics [xmin=0.5,xmax=128.5,ymin=0.5,ymax=128.5] {./images/Q6/Fhat_c_nl-1.png};
\end{axis}
\end{tikzpicture}%}
      \caption{Image $\mathcal{F}(F)$. Origin at the middle. Illustration without the use of the \texttt{log} function.}
      \label{fig:02.Fhat_c_nl}
    \end{figure}
  \end{minipage}
  \hspace{0.05\linewidth}
  \begin{minipage}{0.25\linewidth}
    \begin{figure}[H]
      \scalebox{0.4}{% This file was created by matlab2tikz.
%
%The latest updates can be retrieved from
%  http://www.mathworks.com/matlabcentral/fileexchange/22022-matlab2tikz-matlab2tikz
%where you can also make suggestions and rate matlab2tikz.
%
\begin{tikzpicture}

\begin{axis}[%
width=3.477in,
height=3.477in,
at={(1.205in,0.469in)},
scale only axis,
axis on top,
separate axis lines,
every outer x axis line/.append style={black},
every x tick label/.append style={font=\color{black}},
xmin=0.5,
xmax=128.5,
every outer y axis line/.append style={black},
every y tick label/.append style={font=\color{black}},
y dir=reverse,
ymin=0.5,
ymax=128.5,
hide axis
]
\addplot [forget plot] graphics [xmin=0.5,xmax=128.5,ymin=0.5,ymax=128.5] {./images/Q6/Ghat_c_nl-1.png};
\end{axis}
\end{tikzpicture}%}
      \caption{Image $\mathcal{F}(G)$. Origin at the middle. Illustration without the use of the \texttt{log} function.}
      \label{fig:02.Ghat_c_nl}
    \end{figure}
  \end{minipage}
  \hspace{0.05\linewidth}
  \begin{minipage}{0.25\linewidth}
    \begin{figure}[H]
      \scalebox{0.4}{% This file was created by matlab2tikz.
%
%The latest updates can be retrieved from
%  http://www.mathworks.com/matlabcentral/fileexchange/22022-matlab2tikz-matlab2tikz
%where you can also make suggestions and rate matlab2tikz.
%
\begin{tikzpicture}

\begin{axis}[%
width=3.477in,
height=3.477in,
at={(1.205in,0.469in)},
scale only axis,
axis on top,
separate axis lines,
every outer x axis line/.append style={black},
every x tick label/.append style={font=\color{black}},
xmin=0.5,
xmax=128.5,
every outer y axis line/.append style={black},
every y tick label/.append style={font=\color{black}},
y dir=reverse,
ymin=0.5,
ymax=128.5,
hide axis
]
\addplot [forget plot] graphics [xmin=0.5,xmax=128.5,ymin=0.5,ymax=128.5] {./images/Q6/Hhat_c_nl-1.png};
\end{axis}
\end{tikzpicture}%}
      \caption{Image $\mathcal{F}(H)$. Origin at the middle. Illustration without the use of the \texttt{log} function.}
      \label{fig:02.Hhat_c_nl}
    \end{figure}
  \end{minipage}
\end{minipage}



\subsection{Question 7}

In essence, image $F$ in the spatial domain is a two-dimensional box function:

\begin{equation}
	F(x,y) = \left\{
	\begin{array}{ c l }
	1, & x_1 \leq x \leq x_2
	\\
	0, & everywhere\ else
	\end{array}
	\right.
\end{equation}

Its Fourier transform is:

\begin{multline}
\mathcal{F}(F(x,y)) = \sum_{x=0}^{N-1} \sum_{y=0}^{N-1} f(x,y) \cdot e^{-\dfrac{2 \pi i (xu + yv)}{N}} = \\
\sum_{x=x_1}^{x_2} e^{-\dfrac{2 \pi i x u}{N}} \sum_{y=0}^{N-1} e^{-\dfrac{2 \pi i yv}{N}} = 
\sum_{x=x_1}^{x_2} e^{-\dfrac{2 \pi i x u}{N}} \sum_{y=0}^{N-1} \mathbf{1} \cdot e^{-\dfrac{2 \pi i yv}{N}} = \\
 \delta(v) \cdot \sum_{x=x_1}^{x_2} \cdot e^{-\dfrac{2 \pi i x u}{N}}
\end{multline}

where we exploited the identity

\begin{equation}
	\mathcal{F}(1) = \delta(v)
\end{equation}

Since $\delta(v) = 1$ only where $v=0$, $\mathcal{F}(F(x,y))$ will be non-zero only where $v=0$. Hence, that is why $F$'s Fourier spectrum is concentrated
in the left border. Similarly, the same can be derived for image $G$, but for transposed axes. As for $H$, since the Fourier transform possesses the property of
linearity and $H$ is a linear combination of $F$ and $H$, its Fourier transform is the combination of those of $F$ and $H$.


\subsection{Question 8}

As stated above, application of a logarithm on an image can reveal details in an image to the human eye. A logarithm transformation is an image
enhancement technique used to compress the range of pixel values in an image so that bright regions are finely tuned, but darker ones are tuned coarsely so 
as for differences between pixels to be visible.


\subsection{Question 9}
As exhibited by image $H$ above, its spectrum consists of the superposition of the spectra of images $F,G$. 
This is a direct result of the linearity of the Fourier transform.


In the general case, if $N$ images are combined in a linear way, their collective spectrum will consist of the same combination of their
individual spectra. In formal:

\begin{equation}
	\mathcal{F}(a \cdot f + b \cdot g) = a \cdot \mathcal{F}(f) + b \cdot \mathcal{F}(g)
\end{equation}

where $a,b \in \mathbb{C}$ are constants and $f,g$ functions of real variable(s). 





