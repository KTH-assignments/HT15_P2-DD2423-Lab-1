\section{Section 1.5 - Multiplication}

\subsection{Question 10}

Images \texttt{F .* G} and $\mathcal{F}(\texttt{F .* G})$ are shown in figures \ref{fig:Q10_1} and \ref{fig:Q10_2} respectively.

\begin{minipage}{\linewidth}
  \centering
  \begin{minipage}{0.25\linewidth}
    \begin{figure}[H]
      \scalebox{0.4}{% This file was created by matlab2tikz.
%
%The latest updates can be retrieved from
%  http://www.mathworks.com/matlabcentral/fileexchange/22022-matlab2tikz-matlab2tikz
%where you can also make suggestions and rate matlab2tikz.
%
\begin{tikzpicture}

\begin{axis}[%
width=3.477in,
height=3.477in,
at={(1.205in,0.469in)},
scale only axis,
axis on top,
separate axis lines,
every outer x axis line/.append style={black},
every x tick label/.append style={font=\color{black}},
xmin=0.5,
xmax=128.5,
every outer y axis line/.append style={black},
every y tick label/.append style={font=\color{black}},
y dir=reverse,
ymin=0.5,
ymax=128.5,
hide axis
]
\addplot [forget plot] graphics [xmin=0.5,xmax=128.5,ymin=0.5,ymax=128.5] {./images/Q10/q10_1-1.png};
\end{axis}
\end{tikzpicture}%}
      \caption{Image \texttt{F .* G}.}
      \label{fig:Q10_1}
    \end{figure}
  \end{minipage}
  \hspace{0.05\linewidth}
  \begin{minipage}{0.25\linewidth}
    \begin{figure}[H]
      \scalebox{0.4}{% This file was created by matlab2tikz.
%
%The latest updates can be retrieved from
%  http://www.mathworks.com/matlabcentral/fileexchange/22022-matlab2tikz-matlab2tikz
%where you can also make suggestions and rate matlab2tikz.
%
\begin{tikzpicture}

\begin{axis}[%
width=3.477in,
height=3.477in,
at={(1.205in,0.469in)},
scale only axis,
axis on top,
separate axis lines,
every outer x axis line/.append style={black},
every x tick label/.append style={font=\color{black}},
xmin=0.5,
xmax=128.5,
every outer y axis line/.append style={black},
every y tick label/.append style={font=\color{black}},
y dir=reverse,
ymin=0.5,
ymax=128.5,
hide axis
]
\addplot [forget plot] graphics [xmin=0.5,xmax=128.5,ymin=0.5,ymax=128.5] {./images/Q10/q10_2-1.png};
\end{axis}
\end{tikzpicture}%}
      \caption{Image $\mathcal{F}(\texttt{F .* G})$.}
      \label{fig:Q10_2}
    \end{figure}
  \end{minipage}
    \hspace{0.05\linewidth}
  \begin{minipage}{0.25\linewidth}
    \begin{figure}[H]
      \scalebox{0.4}{% This file was created by matlab2tikz.
%
%The latest updates can be retrieved from
%  http://www.mathworks.com/matlabcentral/fileexchange/22022-matlab2tikz-matlab2tikz
%where you can also make suggestions and rate matlab2tikz.
%
\begin{tikzpicture}

\begin{axis}[%
width=3.477in,
height=3.477in,
at={(1.205in,0.469in)},
scale only axis,
axis on top,
separate axis lines,
every outer x axis line/.append style={black},
every x tick label/.append style={font=\color{black}},
xmin=0.5,
xmax=128.5,
every outer y axis line/.append style={black},
every y tick label/.append style={font=\color{black}},
y dir=reverse,
ymin=0.5,
ymax=128.5,
hide axis
]
\addplot [forget plot] graphics [xmin=0.5,xmax=128.5,ymin=0.5,ymax=128.5] {./images/Q10/q10_3-1.png};
\end{axis}
\end{tikzpicture}%}
      \caption{Image $\mathcal{F}(F) * \mathcal{F}(G)$.}
      \label{fig:Q10_3}
    \end{figure}
  \end{minipage}
\end{minipage}
\\

Since the multiplication in either the spatial or the frequency domain is translated into a convolution in the other,
the same result can be obtained by convolving the Fourier transforms of $F$ and $G$. The result of this operation is
illustrated in figure \ref{fig:Q10_3}.
