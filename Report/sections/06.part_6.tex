\section{Section 1.8 - Information in Fourier phase and magnitude}

\begin{figure}[H]
	% This file was created by matlab2tikz.
%
%The latest updates can be retrieved from
%  http://www.mathworks.com/matlabcentral/fileexchange/22022-matlab2tikz-matlab2tikz
%where you can also make suggestions and rate matlab2tikz.
%
\begin{tikzpicture}

\begin{axis}[%
width=1.601in,
height=1.601in,
at={(2.716in,5.264in)},
scale only axis,
axis on top,
separate axis lines,
every outer x axis line/.append style={black},
every x tick label/.append style={font=\color{black}},
xmin=0.5,
xmax=128.5,
every outer y axis line/.append style={black},
every y tick label/.append style={font=\color{black}},
y dir=reverse,
ymin=0.5,
ymax=128.5,
hide axis
]
\addplot [forget plot] graphics [xmin=0.5,xmax=128.5,ymin=0.5,ymax=128.5] {./images/Q13/q13-1.png};
\end{axis}

\begin{axis}[%
width=1.601in,
height=1.601in,
at={(4.887in,5.264in)},
scale only axis,
axis on top,
separate axis lines,
every outer x axis line/.append style={black},
every x tick label/.append style={font=\color{black}},
xmin=0.5,
xmax=128.5,
every outer y axis line/.append style={black},
every y tick label/.append style={font=\color{black}},
y dir=reverse,
ymin=0.5,
ymax=128.5,
hide axis
]
\addplot [forget plot] graphics [xmin=0.5,xmax=128.5,ymin=0.5,ymax=128.5] {./images/Q13/q13-2.png};
\end{axis}

\begin{axis}[%
width=1.601in,
height=1.601in,
at={(7.059in,5.264in)},
scale only axis,
axis on top,
separate axis lines,
every outer x axis line/.append style={black},
every x tick label/.append style={font=\color{black}},
xmin=0.5,
xmax=128.5,
every outer y axis line/.append style={black},
every y tick label/.append style={font=\color{black}},
y dir=reverse,
ymin=0.5,
ymax=128.5,
hide axis
]
\addplot [forget plot] graphics [xmin=0.5,xmax=128.5,ymin=0.5,ymax=128.5] {./images/Q13/q13-3.png};
\end{axis}

\begin{axis}[%
width=1.601in,
height=1.601in,
at={(2.716in,3.04in)},
scale only axis,
axis on top,
separate axis lines,
every outer x axis line/.append style={black},
every x tick label/.append style={font=\color{black}},
xmin=0.5,
xmax=128.5,
every outer y axis line/.append style={black},
every y tick label/.append style={font=\color{black}},
y dir=reverse,
ymin=0.5,
ymax=128.5,
hide axis
]
\addplot [forget plot] graphics [xmin=0.5,xmax=128.5,ymin=0.5,ymax=128.5] {./images/Q13/q13-4.png};
\end{axis}

\begin{axis}[%
width=1.601in,
height=1.601in,
at={(4.887in,3.04in)},
scale only axis,
axis on top,
separate axis lines,
every outer x axis line/.append style={black},
every x tick label/.append style={font=\color{black}},
xmin=0.5,
xmax=128.5,
every outer y axis line/.append style={black},
every y tick label/.append style={font=\color{black}},
y dir=reverse,
ymin=0.5,
ymax=128.5,
hide axis
]
\addplot [forget plot] graphics [xmin=0.5,xmax=128.5,ymin=0.5,ymax=128.5] {./images/Q13/q13-5.png};
\end{axis}

\begin{axis}[%
width=1.601in,
height=1.601in,
at={(7.059in,3.04in)},
scale only axis,
axis on top,
separate axis lines,
every outer x axis line/.append style={black},
every x tick label/.append style={font=\color{black}},
xmin=0.5,
xmax=128.5,
every outer y axis line/.append style={black},
every y tick label/.append style={font=\color{black}},
y dir=reverse,
ymin=0.5,
ymax=128.5,
hide axis
]
\addplot [forget plot] graphics [xmin=0.5,xmax=128.5,ymin=0.5,ymax=128.5] {./images/Q13/q13-6.png};
\end{axis}

\begin{axis}[%
width=1.601in,
height=1.601in,
at={(2.716in,0.816in)},
scale only axis,
axis on top,
separate axis lines,
every outer x axis line/.append style={black},
every x tick label/.append style={font=\color{black}},
xmin=0.5,
xmax=128.5,
every outer y axis line/.append style={black},
every y tick label/.append style={font=\color{black}},
y dir=reverse,
ymin=0.5,
ymax=128.5,
hide axis
]
\addplot [forget plot] graphics [xmin=0.5,xmax=128.5,ymin=0.5,ymax=128.5] {./images/Q13/q13-7.png};
\end{axis}

\begin{axis}[%
width=1.601in,
height=1.601in,
at={(4.887in,0.816in)},
scale only axis,
axis on top,
separate axis lines,
every outer x axis line/.append style={black},
every x tick label/.append style={font=\color{black}},
xmin=0.5,
xmax=128.5,
every outer y axis line/.append style={black},
every y tick label/.append style={font=\color{black}},
y dir=reverse,
ymin=0.5,
ymax=128.5,
hide axis
]
\addplot [forget plot] graphics [xmin=0.5,xmax=128.5,ymin=0.5,ymax=128.5] {./images/Q13/q13-8.png};
\end{axis}

\begin{axis}[%
width=1.601in,
height=1.601in,
at={(7.059in,0.816in)},
scale only axis,
axis on top,
separate axis lines,
every outer x axis line/.append style={black},
every x tick label/.append style={font=\color{black}},
xmin=0.5,
xmax=128.5,
every outer y axis line/.append style={black},
every y tick label/.append style={font=\color{black}},
y dir=reverse,
ymin=0.5,
ymax=128.5,
hide axis
]
\addplot [forget plot] graphics [xmin=0.5,xmax=128.5,ymin=0.5,ymax=128.5] {./images/Q13/q13-9.png};
\end{axis}
\end{tikzpicture}%
	\caption{The first column features the original images \texttt{phone128, few128, nallo128}. Their respective power spectrum transformation
	\texttt{pow2image} and phase transformation \texttt{randphaseimage} are in the second and third columns respectively.}
	\label{fig:Q13}
\end{figure}


\subsection{Question 13}

Let us consider the one dimension, continuous case. A real signal can be expressed in the form of its inverse Fourier transform:

\begin{multline}
	x(t) = 	\int_{-\infty}^{\infty} X(\omega) e^{j\omega t} d \omega  = 
			\int_{-\infty}^{\infty} |X(\omega)| e^{j \phi} e^{j \omega t} d \omega = 
			\int_{-\infty}^{\infty} |X(\omega)| e^{j(\omega t + \phi)} d \omega
$$			
\label{eq:q13_fourier_decomp}
\end{multline}

where $X(\omega) = |X(\omega)| e^{j \phi}$$, with $$|X(\omega)|$ being the amplitude of the signal in the frequency domain, and $\phi$ its phase.
Equation \ref{eq:q13_fourier_decomp} describes that the amplitude $|X(\omega)|$ is the weight for the contribution of the sinusoid $e^{j\omega t}$ to $x(t)$,
while the $e^{j \phi}$ component determines the phase of this sinusoid with respect to other sinusoids in $x(t)$. 

As seen in figure \ref{fig:Q13} the phase component is responsible for the information that humans can relate to,
since the second column features images whose phase is the same as those of the original images. In contrast, the third column where the amplitude has been
kept the same as that in the first column has no immediate visual information visible.

